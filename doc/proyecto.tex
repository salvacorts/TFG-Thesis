%%%%%%%%%%%%%%%%%%%%%%%%%%%%%%%%%%%%%%%%%
% Short Sectioned Assignment LaTeX Template Version 1.0 (5/5/12)
% This template has been downloaded from: http://www.LaTeXTemplates.com
% Original author:  Frits Wenneker (http://www.howtotex.com)
% License: CC BY-NC-SA 3.0 (http://creativecommons.org/licenses/by-nc-sa/3.0/)
%%%%%%%%%%%%%%%%%%%%%%%%%%%%%%%%%%%%%%%%%

% \documentclass[paper=a4, fontsize=11pt]{scrartcl} % A4 paper and 11pt font size
\documentclass[11pt, a4paper]{book}
\usepackage[T1]{fontenc} % Use 8-bit encoding that has 256 glyphs
\usepackage[utf8]{inputenc}
\usepackage{fourier} % Use the Adobe Utopia font for the document - comment this line to return to the LaTeX default
\usepackage{listings} % para insertar código con formato similar al editor
\usepackage{listings-golang} % import this package after listings
%\usepackage[spanish, es-tabla]{babel} % Selecciona el español para palabras introducidas automáticamente, p.ej. "septiembre" en la fecha y especifica que se use la palabra Tabla en vez de Cuadro
\usepackage[english]{babel}
\usepackage{url} % ,href} %para incluir URLs e hipervínculos dentro del texto (aunque hay que instalar href)
\usepackage{graphics,graphicx, float} %para incluir imágenes y colocarlas
\usepackage[gen]{eurosym} %para incluir el símbolo del euro
\usepackage{cite} %para incluir citas del archivo <nombre>.bib
\usepackage{enumerate}
\usepackage[shortlabels]{enumitem} % Enumerate with a) b) c)...
\usepackage{hyperref}
\usepackage{graphicx}
\usepackage{tabularx}
\usepackage{tabu} % thicker lines on tables
\usepackage{booktabs}
\usepackage{wrapfig} % Wrap images with text
\usepackage{subcaption}
\usepackage{multirow, multicol} % For personas
\usepackage{float}
\usepackage{protobuf/lang}  % include language definition for protobuf
%\usepackage{protobuf/style} % include custom style for proto declarations.
\restylefloat{table} % for personas images

\usepackage[table,xcdraw]{xcolor}
\hypersetup{
	colorlinks=true,	% false: boxed links; true: colored links
	linkcolor=black,	% color of internal links
	urlcolor=cyan		% color of external links
}
\renewcommand{\familydefault}{\sfdefault}
\usepackage{fancyhdr} % Custom headers and footers
\pagestyle{fancyplain} % Makes all pages in the document conform to the custom headers and footers
\fancyhead[L]{} % Empty left header
\fancyhead[C]{} % Empty center header
\fancyhead[R]{Salvador Corts Sánchez} % My name
\fancyfoot[L]{} % Empty left footer
\fancyfoot[C]{} % Empty center footer
\fancyfoot[R]{\thepage} % Page numbering for right footer
%\renewcommand{\headrulewidth}{0pt} % Remove header underlines
\renewcommand{\footrulewidth}{0pt} % Remove footer underlines
\setlength{\headheight}{13.6pt} % Customize the height of the header

\usepackage{titlesec, blindtext, color}
\definecolor{gray75}{gray}{0.75}
\newcommand{\hsp}{\hspace{20pt}}
\titleformat{\chapter}[hang]{\Huge\bfseries}{\thechapter\hsp\textcolor{gray75}{|}\hsp}{0pt}{\Huge\bfseries}
\setcounter{secnumdepth}{4}
\usepackage[Lenny]{fncychap}
\usepackage{color}

\definecolor{brickred}{rgb}{0.8, 0.25, 0.33}
\definecolor{oceanboatblue}{rgb}{0.0, 0.47, 0.75}

\lstset{ % add your own preferences
	basicstyle=\scriptsize,
	breaklines=true,
    basicstyle=\footnotesize,
    keywordstyle=\color{oceanboatblue},
    numbers=left,
    numbersep=5pt,
    showstringspaces=false, 
    stringstyle=\color{brickred},
    tabsize=4,
    escapeinside=\`\`,
    language=Golang % this is it !
}


\begin{document}

	% Plantilla portada UGR
	\input{portada/portada}

	% Plantilla prefacio UGR
	\thispagestyle{empty}

\begin{center}
{\large\bfseries Gocey. Distributed Evolutionary Algorithms on Ephemeral Infrastructure  \\ A Go library to run decentralized evolutionary algorithms powered by a peer-to-peer collaborative science network }\\
\end{center}
\begin{center}
Salvador Corts Sánchez\\
\end{center}

%\vspace{0.7cm}

\vspace{0.5cm}
\noindent{\textbf{Keywords}: \textit{Open Source, Golang, Peer-to-Peer, Distributed systems, Machine Learning, Evolutionary Algorithms}
\vspace{0.7cm}

\noindent{\textbf{Abstract}\\

The amount of internet-capable devices in the world increases each year. Since many of these devices are partially idle all the time, an incredible amount of resources available for computation are being wasted.

The result of this project is Gocey, a Golang library that aims to use these potential resources to build a collaborative science network. This library enables developers to easily implement and deploy scalable and fault-tolerant peer-to-peer systems to run distributed evolutionary algorithms.

Gocey proposes a novel hybrid pool-island distributed evolutionary model where peers in the network can take the role of islands or evaluators. Islands are in charge of applying genetic operators on their individuals and migrating individuals to other islands on the network. Evaluators, on the other hand, are in charge of evaluating the individuals from those islands.

While this model aims to maintains a good balance between exploitation and exploration, Gocey also provides an abstraction on the underlying networking and the distributed system it creates; enabling tech-savvy and non tech-savvy users to collaborate on the system.

	

\cleardoublepage

\begin{center}
	{\large\bfseries Gocey. Algoritmos Evolutivos Distribuidos usando Infraestructura Efímera\\ Una librería de Go para ejecutar algoritmos evolutivos decentralizados utilizando una red peer-to-peer de ciencia colaborativa }\\
\end{center}
\begin{center}
	Salvador Corts Sánchez\\
\end{center}
\vspace{0.5cm}
\noindent{\textbf{Palabras Clave}: \textit{Software Libre, Golang, Peer-2-Peer, Sistemas distribuidos, Aprendizaje Automático, Algoritmos Evolutivos}
\vspace{0.7cm}

\noindent{\textbf{Resumen}\\

La cantidad de dispositivos conectados a internet en el mundo aumenta cada año. Muchos de estos dispositivos no son usados por sus usuarios activamente la mayor parte del tiempo y, por lo tanto, una gran cantidad de recursos disponibles para tareas de computación se desperdician.

El resultado de este proyecto es Gocey, un librería para Golang que aspira a usar estos recursos para construir un sistema de ciencia colaborativa. Esta libraría permite a desarrolladores implementar y desplegar sistemas peer-to-peer escalables y tolerante a fallos, para ejecutar algoritmos evolutivos distribuidos.

Gocey propone un nuevo modelo evolutivo híbrido isla-piscina donde los miembros de la red pueden tomar el rol de islas o evaluadores. Las islas se encargan de aplicar operadores genéticos sobre sus individuos y de migrarlos a otras islas del sistema. Los evaluadores, por otra parte, se encargan de evaluar los individuos de las islas.

Este modelo intenta mantener un buen balance entre exploración y explotación, pero además, Gocey abstrae la red y el sistema distribuido subyacente que crea, de manera que tanto usuarios expertos como no expertos en tecnología pueden colaborar en el sistema.


\cleardoublepage

\chapter*{Acknowledgments}

Thanks to Juan Julián Merelo Guervós, my supervisor, for his help and guidance through the development of this project.

And special thanks to my family who gave me all their love and what they have, to be the person I am today. And to Blanca, for these awesome years that gave me the energy and motivation for this project.

	% Índice de contenidos
	\newpage
	\tableofcontents

	% Índice de imágenes y tablas
	\newpage
	\listoffigures

	% Si hay suficientes se incluirá dicho índice
	\listoftables 
	\newpage

	% Introducción 
	\input{secciones/01_introduccion}

	% Descripción del problema y hasta donde se llega
	\input{secciones/02_descripcion}

	% Estado del arte
	% 	1. Crítica al estado del arte
	% 	2. Propuesta
	\input{secciones/03_estado_del_arte}
	
	\input{secciones/04_planificacion}

	% Desarrollo bajo sprints: 
	% 	1. Permitir registros y login de usuarios
	% 	2. Desarrollo del sistema de incidencias
	% 	3. Desarrollo del sistema de denuncias administrativas y accidentes
	% 	4. Desarrollo del sistema de croquis
	%   5. Instalación de la aplicación de manera automática
	\input{secciones/06_implementacion}

	% Resultados
	\input{secciones/07_results}
	
	% Presupuesto
	\chapter{Development Time and Budget}

In this chapter we will analyze the time needed to develop this project and set a price for it. Then, we will analyze different alternatives to deploy this system and the budget we would need to do so.

\section{Project Billing}

We will calculate the value of this project in terms of the time needed to develop it. Thanks to have been using \textit{Git} as our version control system, we have been keeping track of the time expended on this project indirectly.

We will use \textit{Git Hours} \cite{git-hours} to calculate the time spent based on the time-stamp of the commits pushed to our repository and the difference in time between them. The algorithm iterates through all commits on a repository and works as follows:

\begin{enumerate}
	\item If the difference of the time-stamp between two commits is equal or smaller than a given threshold (in our case two hours), group them into the same development session. Otherwise, end the current development session and create a new session starting with the second commit.
	
	\item Since before the first commit of a session, there is some time needed to make the changes from it; add three extra hours to the coding session.
	
	\item Aggregate time from all coding sessions.
\end{enumerate}

In our case, we invested approximately 170, hours to develop this project, and 127 for this document. Since the average Junior Software Engineer salary in Spain is 20,456€ \cite{payscale}; working 5 days a week, 8 hours per day, it equal to 9.83€ per hour. Hence, an estimated value for our project would be 1671.1€, or 2919.51€ taking into account this document as well.


\section{Deployment Budget}

Even though this project can be ran on modest machines along with other processes in the background, a developer might want to dedicate specific resources to run several nodes. We will analyze the usage cost of the system based on the number of nodes and the major infrastructure providers: Google, Amazon and Microsoft.

Since we would need at least two clients per island to outperform a sequential model, at least three nodes must be running on the system. This will be our minimum operational cost.

\paragraph*{Google Cloud.} Using \textit{Compute Engine} (\textit{n1-standard-1}); a virtual machine running Ubuntu, with 1 core, and 3.75 GB of RAM. Running a node would cost 24.27 US\$ per month. Therefore, the minimum operational cost is 72.81 US\$. \cite{gcp-calculator}

\paragraph*{Amazon Web Services.} Using an \textit{EC2 Instance} (t3a.medium), with Linux, 1 core, and 4 GB of RAM would cost 17.23 US\$ per month per node. Minimum operational cost of 51.69 US\$ per month. \cite{aws-calculator}

\paragraph*{Azure.} Using a standard virtual machine (\textit{A1}), with Ubuntu, 1 core, and 1.75GB of RAM would have a monthly cost of 22.63 US\$ per node in the system. Minimum operational cost of 67.89 US\$. \cite{azure-calculator}

\paragraph*{}
As can be seen, the monthly cost of the system is really low, and even more if we taken into account that different people can collaborate with each other and share operational cost.

	% Conclusiones y Trabajos futuros
	\chapter{Conclusions and Future Work}

We have presented the design, implementation, and results of Gocey; a library to easily implement collaborative distributed and decentralized evolutionary algorithms that abstracts the underlying networking and distributed system with a scalable and fault-tolerant peer-to-peer design.

Gocey also provides an extensible design and compatibility with the \textit{eaopt} library \cite{eaopt}. While many evolutionary algorithms libraries only support the classic mutation and crossover operators out of the box, this project enables the user of the library to use as many custom operators as desired while allowing they to use all the already existing mutation and crossover operators from the \textit{eaopt} library.

The proposed distributed hybrid pool-island evolutionary model looks like a promising model for problems where the evaluation operator is computationally expensive, problems that requires high exploratory capabilities, and for multi-objective problems where several parameters of a solution have to be optimized. In the last case, several clusters can be created where nodes from each cluster will apply genetic operators that modifies a specific gene of the solution representation.

Even though the use of experimental features from Go's Web Assembly support led us to unexpected constraints and challenges during the development of the project, we were able to address them by looking for alternatives such as the web-sockets listener to serve gRPC requests for browser-based clients. As the Web Assembly support for Golang increases, the performance achieved by this project will increase without having to change much of its code-base.

Of course, Gocey has plenty of room for improvements, and we would like to mention some of them. With regard to the genetic operators that are applied over an island in our design, new selection methods can be implemented so the user of the library has the ability to choose other than the tournament of k individuals. Also, it would be nice to enable the user of the library to decide which operators do the clients apply, so they do not just evaluate individuals from a pool. 

In order to reduce latency between clients and islands, a batching mechanism can be implemented where instead of asking for just one individual on each \textit{BorrowIndividual} request, the client can ask for as many individuals as can fit into a single TCP packet.

With regard to the current client implementation, it could be interesting to design a mechanism where clients in the same network can collaborate between each other like on a cellular distributed evolutionary model. 

Also, having support for Android devices would increase the number of potential collaborators in our platform. note that since we are using Protocol Buffers and gRPC, exchange of information between the Android client and the current implementation should be trivial. 

Besides, even though by having browser support we achieved zero installation requirements for collaborators, at cost of installation requirements, a web extension would increase the chances of a collaborator to start collaborating again every time they open their browser.

The user experience for browser-based collaborators can be further improved. While with the current implementation the user only gets feedback about their self and the island he is evaluating individuals from, displaying a ranking of collaborators in terms of the total number of evaluated individuals could improve user engagement by introducing some competition between collaborators. 

As stated in the previous chapter, to increase the chances of new users using our platform, it could be interesting to have a catalog where researchers can share their clusters and where collaborators can look for problems they might be willing to collaborate to.

All in all, this work was not only a engineering project but also a great opportunity to work on distributed systems and artificial intelligence; two of the fields of Computer Science that inspire me the most. Furthermore, it was also an opportunity to work with really talented people, and to apply the knowledge acquired in these four years at University and developing my side-projects.

	
	\newpage
	\bibliography{bibliografia}
	\bibliographystyle{plain}
	
\end{document}

