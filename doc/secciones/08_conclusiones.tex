\chapter{Conclusions and Future Work}

We have presented the design, implementation, and results of Name\_Of\_Project; a library to easily implement collaborative distributed and decentralized evolutionary algorithms that abstracts the underlying networking and distributed system with a scalable and fault-tolerant peer-to-peer design.

Name\_Of\_Project also provides an extensible design and compatibility with the \textit{eaopt} library \cite{eaopt}. While many evolutionary algorithms libraries only support the classic mutation and crossover operators out of the box, this project enables the user of the library to use as many custom operators as desired while allowing they to use all the already existing mutation and crossover operators from the \textit{eaopt} library.

The proposed distributed hybrid pool-island evolutionary model looks like a promising model for problems where the evaluation operator is computationally expensive, problems that requires high exploratory capabilities, and for multi-objective problems where several features of a solution have to be optimized. In the last case, several clusters can be created where nodes from each cluster will apply genetic operators that modifies a specific gene of the solution representation.

Even though the use of experimental features from Go's Web Assembly support led us to unexpected constraints and challenges during the development of the project, we were able to address them them by looking for alternatives such as the web-sockets listener to serve gRPC requests for browser-based clients. As the Web Assembly support for Golang increases, the performance achieved by this project will increase without having to change much of its code-base.



\paragraph*{Future Work}. This project can be extended to ...



All in all, this work was not only a research project, but also a great opportunity to work on distributed systems and artificial intelligence; two of the fields of Computer Science that excite me the most. Furthermore, it was also an opportunity to work with really talented people, and to apply many of my engineering capabilities acquired in these four years at University and developing my own side-projects.