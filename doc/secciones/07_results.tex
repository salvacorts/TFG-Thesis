\chapter{Experiments and Results}

In this chapter we will iterate through each of our initial objectives, analyzing their accomplishment and the obtained results.

\section{Objective 1. Maximize engagement of contributors}

\paragraph*{Ease of installation}
\paragraph*{Browser-based node}
\paragraph*{Visual feedback of execution on browser-based clients} 

Final discussion about the achievement of this objective



\section{Objective 2. Fault-tolerant design capable of working with heterogeneous and ephemeral nodes}

Where am I in the CAP theorem?

\paragraph*{Heterogeneous Nodes}
\begin{itemize}
	\item Multi-Platform Code - list of supported go architectures and WASM
\end{itemize}

\paragraph*{Fault-tolerance}
\begin{itemize}
	\item Decentralized design - no single point of failure
	\item Partition tolerance - example of nodes running on different sub-nets (use Hamid's tool?).
\end{itemize}

\paragraph*{Ephemeral Nodes}
\begin{itemize}
	\item Failure detection system - detect failing nodes with low band-wight usage and decentralized way.
	\item Nodes join system - How easy is it? how reliable is it? - How easy is to create a new cluster?
\end{itemize}

Final discussion about the achievement of this objective



\section{Objective 3. Achieve higher performance level than a non-distributed implementation}

\paragraph*{Evaluations Throughput.} Number of evaluations per second - chart of evaluations per second vs number of clients
\paragraph*{Exploratory capabilities?}
\paragraph*{Exploitation capabilities?}
\paragraph*{Comparison with G-Prop} - Use saved logs - Add/Remove neurons if not achieve better results.

Final discussion about the achievement of this objective